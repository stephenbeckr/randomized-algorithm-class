% Some commonly used latex settings. Stephen Becker, July 2013
\pagestyle{plain}
%--------------
\newtheorem{theorem}{Theorem}[section]
\newtheorem{lemma}[theorem]{Lemma}
\newtheorem{corollary}[theorem]{Corollary}
\newtheorem{proposition}[theorem]{Proposition}
\newtheorem{definition}[theorem]{Definition}
\newtheorem{conjecture}[theorem]{Conjecture}
\newtheorem{problem}[theorem]{Problem} 
\newtheorem{fact}[theorem]{Fact} % added Jan 2014
\newtheorem{assumption}[theorem]{Assumption} % added Jan 2014
\newtheorem{remark}[theorem]{Remark}
\newtheorem{remarks}[subsection]{Remarks}
\newtheorem{example}[subsection]{Example}
%\newtheorem{example}[theorem]{Example}
%\floatname{algorithm}{Listing}
%\numberwithin{equation}{section}  % For now, commenting this out since I do NOT want eq numbers like (0.1)
% Theorems
%\newtheorem{theorem}{Theorem}
%\newtheorem{lemma}{Lemma}
%\newtheorem{remark}{Remark}
%\newtheorem{corollary}{Corollary}%[section]
%\newtheorem{proposition}{Proposition}%[section]
%\newtheorem{definition}{Definition}%[section] % number this the same as theorem and lemma


%% commenting
% Affect margins:
%\setlength{\marginparwidth}{1.2in}
\setlength{\marginparwidth}{.8in}
\let\oldmarginpar\marginpar
\renewcommand\marginpar[1]{\-\oldmarginpar[\raggedleft\footnotesize #1]%
{\raggedright\footnotesize #1}}

% macros for the outline
\newcommand{\todo}{{\bf \textcolor{red}{TODO} }}
\newcommand{\TODO}[1]{{\bf TODO: #1}}
\newcommand{\red}{\textcolor{red}}
\newcommand{\note}[1]{{\bf [{\em Note:} #1]}}

% Editing commands
\newcommand{\fix}[1]{\textcolor{red}{#1}}
\usepackage[normalem]{ulem} % for sout, not needed for final version
\newcommand{\add}[1]{\textcolor{blue}{#1}}
\newcommand{\new}[1]{\textcolor{blue}{#1}} % synonym
\newcommand{\del}[1]{{\color{Bittersweet}\sout{#1}}}
\newcommand{\remove}[1]{{\color{Bittersweet}\sout{#1}}} % synonym

% Better, use e.g., (with comma)
%\newcommand\eg{e.g.\xspace}
%\newcommand\ie{i.e.\xspace}





\newcommand{\Id}{\text{\em I}}
\newcommand{\OpId}{\mathcal{I}}

% -- Operators --
%First of all, one must of course recall that \operatorname and \DeclareMathOperator are provided by the amsopn package, which is automatically loaded by amsmath, but is also available standalone
\DeclareMathOperator{\dom}{dom} 
\DeclareMathOperator{\vect}{vec}            % vec(X) = X(:) in matlab notation
\DeclareMathOperator{\VEC}{vec}            % vec(X) = X(:) in matlab notation
\DeclareMathOperator{\mat}{mat}             % mat(x) = reshape(x,N,N)
\DeclareMathOperator{\prox}{prox}           
\DeclareMathOperator{\tr}{trace}
\DeclareMathOperator{\logdet}{log det}
%\newcommand{\sgn}{\textrm{sgn}}
%\newcommand{\sign}{\textrm{sgn}}  or instead \operatorname
\DeclareMathOperator{\shr}{shrink}
\DeclareMathOperator{\shrink}{shrink}
\DeclareMathOperator{\trunc}{trunc}
\DeclareMathOperator{\range}{range}
\DeclareMathOperator{\rank}{rank}
\DeclareMathOperator{\diag}{diag}
\DeclareMathOperator{\trace}{trace}
\DeclareMathOperator{\supp}{supp}
\DeclareMathOperator*{\argmax}{argmax}        % puts subscripts in the right place
\DeclareMathOperator*{\argmin}{argmin}
\DeclareMathOperator*{\minimize}{minimize}
\DeclareMathOperator*{\maximize}{maximize}
% -- Misc --
\newcommand\thalf{{\textstyle\frac{1}{2}}}
\newcommand{\eps}{\varepsilon}
\newcommand{\e}{\mathrm{e}}
\renewcommand{\i}{\imath}
%\newcommand{\bmat}[1]{\begin{bmatrix} #1 \end{bmatrix}}
\newcommand{\smax}{\sigma_{\max}}
\newcommand{\smin}{\sigma_{\min}}
%\newcommand{\T}{*}                           % (see also \transp, \adj below)
\newcommand{\T}{T}                            % for the adjoint/transpose
\newcommand{\transp}{T}
\newcommand{\adj}{*}
\newcommand{\psinv}{\dagger}
% -- Mathbb --
\newcommand{\R}{\mathbb{R}}
\newcommand{\RR}{\mathbb{R}}
\newcommand{\Rn}{\R^{n}}
\newcommand{\Rmn}{\R^{m \times n}}
\newcommand{\Rnn}{\R^{n \times n}}
\newcommand{\Rmm}{\R^{m \times m}}
\newcommand{\C}{\mathbb{C}}
\newcommand{\Z}{\mathbb{Z}}
\newcommand{\HH}{\mathcal{H}}                  % for Hilbert space (\H already defined).
\newcommand{\EE}{\operatorname{\mathbb{E}}}   % for probability and expectations
\newcommand{\E}{\operatorname{\mathbb{E}}} % is operatorname necessary?
\renewcommand{\P}{\operatorname{\mathbb{P}}}  % for probability
% -- Mathcal --
\newcommand{\id}{\mathcal{I}} % identity operator
\newcommand{\AAA}{\ensuremath{\mathcal{A}}}   % generic linear operator
\newcommand{\cA}{\ensuremath{\mathcal{A}}}    % generic linear operator
\newcommand{\K}{\ensuremath{\mathcal{K}}}     % cone
\newcommand{\cK}{\ensuremath{\mathcal{K}}}    % cone
\newcommand{\proj}{\ensuremath{\mathcal{P}}}  % Projection
\newcommand{\PP}{\operatorname{\mathcal{P}}}  % for projections
\newcommand{\lag}{\ensuremath{\mathcal{L}}}   % Lagrangian
\renewcommand{\L}{{\mathcal L}}
\newcommand{\N}{{\mathcal{N}}}                % for normal N(0,1) variables...
\newcommand{\order}{\mathcal{O}}              % big O notation
% -- Text shortcuts --
\newcommand{\st}{\ensuremath{\;\text{such that}\;}}
%\newcommand{\st}{\text{subject to}}
\newcommand{\gs}{g_\text{sm}}             % smooth part of dual objective


%  -- To get the ones vector to look nice (without using the bbold package)
\newcommand{\bbfamily}{\fontencoding{U}\fontfamily{bbold}\selectfont}
\newcommand{\textbb}[1]{{\bbfamily#1}}
\DeclareMathAlphabet{\mathbbb}{U}{bbold}{m}{n}
\newcommand{\ones}{\mathbbb 1}                % ones vector 

% -- For := type stuff --
%\newcommand{\defeq}{\mathrel{\mathop:}=}      % for definitions, e.g. z := y + 3
%\newcommand{\defeq}{\triangleq}               %   another alternative
%\newcommand{\defeq}{\equiv}                   %   another alternative
\newcommand{\defeq}{\stackrel{\text{\tiny def}}{=}}  %   another alternative
%\newcommand{\defeq}{\stackrel{\text{\tiny def}}{\hbox{\equalsfill}}}  % another alternative, doesn't work


% -- Inner products and norms --
\newcommand{\<}{\langle}
\renewcommand{\>}{\rangle}
\newcommand{\restrict}[1]{\big\vert_{#1}}
% If using < x | y > or { x | x < 0 }
%http://tex.stackexchange.com/questions/498/mid-vertical-bar-vert-lvert-rvert-divides
%use \mid not | (bar, bracket) for inner products and such.
%\newcommand{\iprod}[2]{\left\langle #1 , #2 \right\rangle}
\newcommand{\iprod}[2]{\left\langle #1,\,#2 \right\rangle}
\newcommand{\iprodMed}[2]{\Bigl\langle #1 , #2 \Bigr\rangle}
\newcommand{\scal}[2]{\left\langle{#1},\,{#2}\right\rangle}
\newcommand{\norm}[1]{{\left\lVert{#1}\right\rVert}}
\newcommand{\dist}[2]{\left\| #1 - #2 \right\|_2}
\newcommand{\vectornormbig}[1]{\big\|#1\big\|}
\newcommand{\vectornormmed}[1]{\big\|#1\big\|}


% Linear algebra macros
%\newcommand{\vct}[1]{\bm{#1}}
%\newcommand{\mtx}[1]{\bm{#1}}
\newcommand{\vct}[1]{{#1}}
\newcommand{\mtx}[1]{{#1}}
%\newcommand{\mtx}[1]{\mathsfsl{#1}}
\renewcommand{\vec}[1]{{\boldsymbol{#1}}}



% -- use amsthm instead --
%\def \endprf{\hfill {\vrule height6pt width6pt depth0pt}\medskip}
%\newenvironment{proof}{\noindent {\bf Proof} }{\endprf\par}
%\newcommand{\qed}{{\unskip\nobreak\hfil\penalty50\hskip2em\vadjust{}
           %\nobreak\hfil$\Box$\parfillskip=0pt\finalhyphendemerits=0\par}}

